\documentclass[12pt,letterpaper]{article}
\usepackage{graphicx,textcomp}
\usepackage{natbib}
\usepackage{setspace}
\usepackage{fullpage}
\usepackage{color}
\usepackage[reqno]{amsmath}
\usepackage{amsthm}
\usepackage{fancyvrb}
\usepackage{amssymb,enumerate}
\usepackage[all]{xy}
\usepackage{endnotes}
\usepackage{lscape}
\newtheorem{com}{Comment}
\usepackage{float}
\usepackage{hyperref}
\newtheorem{lem} {Lemma}
\newtheorem{prop}{Proposition}
\newtheorem{thm}{Theorem}
\newtheorem{defn}{Definition}
\newtheorem{cor}{Corollary}
\newtheorem{obs}{Observation}
\usepackage[compact]{titlesec}
\usepackage{dcolumn}
\usepackage{tikz}
\usetikzlibrary{arrows}
\usepackage{multirow}
\usepackage{xcolor}
\newcolumntype{.}{D{.}{.}{-1}}
\newcolumntype{d}[1]{D{.}{.}{#1}}
\definecolor{light-gray}{gray}{0.65}
\usepackage{url}
\usepackage{listings}
\usepackage{color}

\definecolor{codegreen}{rgb}{0,0.6,0}
\definecolor{codegray}{rgb}{0.5,0.5,0.5}
\definecolor{codepurple}{rgb}{0.58,0,0.82}
\definecolor{backcolour}{rgb}{0.95,0.95,0.92}

\lstdefinestyle{mystyle}{
	backgroundcolor=\color{backcolour},   
	commentstyle=\color{codegreen},
	keywordstyle=\color{magenta},
	numberstyle=\tiny\color{codegray},
	stringstyle=\color{codepurple},
	basicstyle=\footnotesize,
	breakatwhitespace=false,         
	breaklines=true,                 
	captionpos=b,                    
	keepspaces=true,                 
	numbers=left,                    
	numbersep=5pt,                  
	showspaces=false,                
	showstringspaces=false,
	showtabs=false,                  
	tabsize=2
}
\lstset{style=mystyle}
\newcommand{\Sref}[1]{Section~\ref{#1}}
\newtheorem{hyp}{Hypothesis}

\title{Problem Set 1}
\date{Due: January 29, 2020}
\author{QTM 200: Applied Regression Analysis}

\begin{document}
	\maketitle
	
	\section*{Instructions}
	\begin{itemize}
		\item Please show your work! You may lose points by simply writing in the answer. If the problem requires you to execute commands in \texttt{R}, please include the code you used to get your answers. Please also include the \texttt{.R} file that contains your code. If you are not sure if work needs to be shown for a particular problem, please ask.
		\item Your homework should be submitted electronically on the course GitHub page in \texttt{.pdf} form.
		\item This problem set is due at the beginning of class on Wednesday, January 22, 2020. No late assignments will be accepted.
		\item Total available points for this homework is 100.
	\end{itemize}
	
	\vspace{1cm}
	\section*{Question 1 (25 points)}
	
	A private school counselor was curious about the average of IQ of the students in her school and took a random sample of 25 students' IQ scores. The following is the data set:
	\vspace{.5cm}
	
	
	
	\vspace{.5cm}
	
	\noindent Find a 90\% confidence interval for the student IQ in the school assuming the population of IQ from which our random sample has been selected is normally distributed.
	
	\lstinputlisting[language=R, firstline=40, lastline=69]{PS01-Answer_R.R}   
	
	\vspace{1cm}
	\section*{Question 2 (25 points)}
	A private school counselor was curious  whether  the average of IQ of the students in her school is higher than the average IQ score 100 among all the schools in the country. She took a random sample of 25 students' IQ scores. The following is the data set:
	\vspace{.5cm}
	\lstinputlisting[language=R, firstline=46, lastline=46]{PS1.R}  
	\vspace{.5cm}
	
	\noindent Conduct a test with 0.05 significance level assuming the population of IQ from which our random sample has been selected is normally distributed. 
	
	\lstinputlisting[language=R, firstline=70, lastline=103]{PS01-Answer_R.R}  
	
	\vspace{1cm}
	\section*{Question 3 (50 points)}
	
	\noindent Researchers are curious about what affects the education expenditure on public education. The following is availabe variables in a data set about the education expenditure. \\

	
	\vspace{.5cm}
	
	
	\begin{tabular}{r|l}
		\texttt{State} &\emph{50 states in US} \\
		\texttt{Y} & \emph{per capita expenditure on public education}\\
		\texttt{X1} &\emph{per capita personal income} \\
		\texttt{X2} &  \emph{Number of residents per thousand under 18 years of age}\\
		\texttt{X3} &  \emph{Number of people per thousand residing in urban areas} \\
		\texttt{Region} &  \emph{1=Northeast, 2= North Central, 3= South, 4=West} \\
	\end{tabular}
	
	\vspace{.5cm}
	\noindent Explore the \texttt{expenditure} data set and import data into \texttt{R}.
	\vspace{.5cm}
	\lstinputlisting[language=R, firstline=54, lastline=54]{PS1.R}  
	\vspace{.5cm}
	\begin{itemize}
		
		\item
		Please plot the relationships among \emph{Y}, \emph{X1}, \emph{X2}, and \emph{X3}? What are the correlations among them (you just need to describe the graph and the relationships among them)?
		
			\vspace{.5cm}
		
		   \begin{figure}\centering
			\caption{Relationship between per capital personal income and per capita expenditure on public education}
			\includegraphics[width=0.75\textwidth]{plot1.pdf}
		   \end{figure}
	
			\vspace{.5cm}
			
			\begin{figure}\centering
				\caption{Relationship between number of residents per thousand under 18 years of age and per capital expenditure on public education}
				\includegraphics[width=0.75\textwidth]{plot2.pdf}
			\end{figure}
		
				\vspace{.5cm}
				
			\begin{figure}\centering
			\caption{Relationship between number of people per thousand residing in the urban areas and per capital expenditure on public education}
			\includegraphics[width=0.75\textwidth]{plot3.pdf}
		   \end{figure}
	
	
	\lstinputlisting[language=R, firstline=118, lastline=137]{PS01-Answer_R.R}  
				\vspace{.5cm}
				
			
		
		Please plot the relationship between \emph{Y} and \emph{Region}? On average, which region has the highest per capita expenditure on public education?\\
		
		
		
			\lstinputlisting[language=R, firstline=138, lastline=149]{PS01-Answer_R.R} 
			
						\vspace{.5cm}
							\vspace{.5cm}
			
			 \begin{figure}\centering
				
				\caption{Relationship between Regions and per capital expenditure on public education}
				\includegraphics[width=0.75\textwidth]{plot4.pdf}
			\end{figure}
		 
        
    	
    	    	
		
		
			
		
	
	


    
	
		
		Please plot the relationship between \emph{Y} and \emph{X1}? Describe this graph and the relationship. Reproduce the above graph including one more variable \emph{Region} and display different regions with different types of symbols and colors.\\
		
			\lstinputlisting[language=R, firstline=150, lastline=161]{PS01-Answer_R.R}  
	      	\begin{figure}\centering
	      	
			\caption{Relationship between per capital personal income and per capital expenditure on public education}
			\includegraphics[width=0.75\textwidth]{plot5.pdf}
		    \end{figure}
		
		\vspace{.5cm}
		
	     	\begin{figure}\centering
			\caption{Relationship between per capital personal income and per capital expenditure on public education with Region on the side }
			\includegraphics[width=0.75\textwidth]{plot6.pdf}
		    \end{figure}

	\end{itemize}
	
\end{document}

\\
